\documentclass[12pt]{amsart}
\usepackage{amssymb}
\usepackage{latexsym}
\title{The trap for \LaTeX\ convertors}
\author{Sergej V. Znamenskij}
\begin {document}
\maketitle

This file contains absolutely legal and ordinary
mathematical constructions in \LaTeX\ text to facilitate visual
detection of image alignment algorithm flaws in \LaTeX\ to HTML conversion.

\section {Line alignment problem}
The problem comes with the large height and (or) depth formulae. 
Usually such a formula as $x_s^{\int\limits_a^{N^1}\, dx}$ for example has
larger height than depth. As a corollary, the convertor may add improper large gap (vskip) before the next
line. The case of equal height and width $\left(x_s^{\int\limits_a^{N^1}\, dx}\right)$ is more controlable, 
as far as we do not try to get images scaled together with text changing.
The most rare and difficult situation is a formula with a large depth, but a very small height: 
$y=1-\dfrac{\alpha}{1-\dfrac{2}{3-\dfrac{4}{5-x}}}$.
I will appreciate any idea to keep proper interline distance in such a case without 
additional vskip over this formula though I suppose it to be impossible: I failed to find a hint in html 
specifications how to force browser to leave a proper place lower image without extra gap over it. 

\section {Inline alignment problems}All dots in the line are to be exactly aligned for any browser and selected text size.

.$.\sum\limits^n1.$.$.\prod\limits_n2.$.$.\dfrac{\alpha\otimes\mu}{\gamma}.$.$.1.$.$.\dfrac {\rightharpoonup}{A\otimes B}.$.

The simple convertion algorithm can fail due to the following reasons:
\begin {itemize} 
\item HTML lacks opportunity to reproduce \LaTeX\ layout directly;
\item Some kind of HTML markup has different visual representation in different browser versions;  
\item Usable image conversion libraries and tools do not care reference point position in the image while cropping;
\item \LaTeX\ box dimensions (width, height and depth) \textbf{usually differs} from its visual width, height, or depth. 
\end {itemize}

Test of Figure environment:
\begin{figure}
\[ a = \int exp{x} \,\mathrm{d}x \]
\end{figure}

\end {document}
